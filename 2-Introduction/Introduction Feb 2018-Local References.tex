%% LyX 2.3.0rc2 created this file.  For more info, see http://www.lyx.org/.
%% Do not edit unless you really know what you are doing.
\documentclass[oneside,american,english,oldfontcommands,oneside, article, extrafontsizes, 10pt, latin9, utf8,main=american]{memoir}
\usepackage[T1]{fontenc}
\usepackage{geometry}
\geometry{verbose,tmargin=1in,bmargin=1in,lmargin=1in,rmargin=1in}
\setcounter{secnumdepth}{3}
\setcounter{tocdepth}{3}
\synctex=-1
\usepackage{color}
\usepackage{babel}
\usepackage{verbatim}
\usepackage{enumitem}
\DoubleSpacing
\usepackage[unicode=true,
 bookmarks=true,bookmarksnumbered=false,bookmarksopen=false,
 breaklinks=false,pdfborder={0 0 0},pdfborderstyle={},backref=false,colorlinks=true]
 {hyperref}
\hypersetup{pdftitle={Rethinking Land Rent: Towards a coherent, pluralist political economy},
 pdfauthor={Marshall M.A. Feldman},
 pdfsubject={Land Rent},
 pdfkeywords={Geographical Political Economy, Circuit of Capital, Land Rent}}

\makeatletter
%%%%%%%%%%%%%%%%%%%%%%%%%%%%%% Textclass specific LaTeX commands.
\newlength{\lyxlabelwidth}      % auxiliary length 

\@ifundefined{date}{}{\date{}}
%%%%%%%%%%%%%%%%%%%%%%%%%%%%%% User specified LaTeX commands.
\input{../general-preamble}
% Master preamble used for master document
% Based on Aaron Defazio's "WRITING YOUR THESIS IN LYX — A SETUP GUIDE" @ http://www.aarondefazio.com/tangentially/?p=19

% My Memoir customizations
%\usepackage[draft]{myMemoir}% See myMemoir for other options
\usepackage{myMemoir}

\author{\myName\thanks{\myContact}}
%\author{\myName}

% Customizations for this paper
\title{Rethinking Land Rent}
\subtitle{Towards a coherent, pluralist political economy}
\date{\today}% Change to an actual date if you wish
\event{Paper prepared for\\The American Association of Geographers Annual Meeting\\April 10-14, 2018\\New Orleans, LA, USA}
%\acknowledge{Thanks to Antoine Godin, Russell Standish, and Steve Keen for helping me use the software packages they developed, including at times revising the software to better meet my needs.}

% Revision information.
%\revision{6.0} -- Not implemented w/ this version of myMemoir
\graphicspath{{../images/}{../graphs}} % Override the one in the general preamble.

%%%% The following are preliminary customizations for LyX 2.3 built-in BibLaTeX that give, IMHO,
%%%% a much better format than the default CMS author-date.
%%%% See Section 3.11 of the Biblatex manual for these and (so many) other commands.
%%%% 
%%%% Once satisfied that these things work, put them in the general start-up file.
%\renewcommand{\nameyeardelim}{\addspace}
%\renewcommand{\postnotedelim}{\addcolon\space}

\makeatother

\usepackage[style=chicago-authordate,doi=only,bookpages=false,isbn=false,compresspages]{biblatex}
\addbibresource{\string"/Users/marsh/OneDrive/Documents/Research-King Mac/Political Economy/Rethinking Rent/references.bib\string"}
\begin{document}
\begin{comment}
With myMemoir, don't put title, etc. in LyX.
\end{comment}


\section{\label{sec:Introduction}Introduction}

After over two decades of relative neglect, land-rent theory may be
experiencing a revival. Several recent essays find existing theory
deficient, give reasons for its current condition, suggest ways forward,
and generally urge its renewal. These similarities notwithstanding,
they differ in their explanations for rent theory's current state
and their suggestions for renewal. Despite these differences they
share certain preconceptions, some of which I challenge here.

Perhaps the most fundamental of these is the near universal tendency
to treat land rent primarily -- almost solely -- from the standpoint
of what \textcite[27-87]{bhaskar2014thepossibility} calls ``the
duality of praxis'' while ignoring what I shall call ``societal
duality.''\footnote{The limitations of language pose a real obstacle here. \citeauthor{bhaskar2014thepossibility}
actually counterpoises ``Society'' and ``Individuals,'' but this
is problematic for several reasons. When he discusses ``society''
he is really discussing social structure. The term ``society'' itself
commonly describes situations that can be much more ad hoc, informal,
and intimate: for example, a small group gathering around the water
cooler in an office and discussing the Academy Awards can be described
as being social and participating in society. But although social
structure is present in this setting, describing the setting as ``society''
would be a misuse of how \citeauthor{bhaskar2014thepossibility} uses
the term. ``Social structure,'' on the other hand, is also problematic
because it suggests an ahistorical ``structuralism'' that reifies
society as structural determinations sweep along lifeless tr�ger who
are just the bearers of social structure \parencite{thompson1978thepoverty}.
The notion I am getting at is much more satisfactorily labelled ``societal''
if one understands this term as connoting that which is large-scale
and society-wide \parencite{etzioni1968theactive}. So instead of
\citeauthor{bhaskar2014thepossibility}'s (\citeyear[38]{bhaskar2014thepossibility})
``duality of structure,'' I use ``societal duality.'' Whichever
term one uses, what \textcite[37, original emphasis]{bhaskar2014thepossibility}means
by it is that ``society is both the ever-present \emph{condition}
(material cause) and the continually repro-duced \emph{outcome} of
human agency.

But this does not get us entirely out of the woods. Bhaskar's use
of ``Individuals'' is even more problematic. While it is true that
some social scientists (and politicians) reduce the human world to
asocial, ahistorical ``individuals'' (e.g. in methodological individualism),
thankfully such practice is becoming increasingly rare. Certainly,
outside of perhaps mainstream/neoclassical economics, the literature
on land rent is not overly concerned with individuals. But it does
emphasize firms, industries, government agencies and policies, localities,
and social actors, even though the latter often stand in as personifications
for broader social categories (``landlords,'' ``tenants,'' etc.).
In relation to individuals, \textcite[37-38, original emphasis]{bhaskar2014thepossibility}
uses ``praxis'' to denote conscious, intentional human activity,
and it too has a dual character in that ``praxis is both work, that
is conscious \emph{production}, and (normally unconscious) \emph{reproduction}
of the conditions of production, that is society.''

While the literature on land rent in economic geography does somewhat
address this dual character, it also goes beyond land rent as the
praxis of individuals, organizations, and the like. Nonetheless, it
rarely gets to the societal level. And there simply is no single English
term that adequately captures the domain in which the land-rent literature
operates.

One might think that ``microeconomic,'' ``macroeconomic,'' and
``meso-'' levels would work, but such terms bring their own problems.
One is that the first two of these are strongly associated with mainstream
economics, which economic geography has largely rejected over the
past forty-five years. Using these terms now could suggest bringing
mainstream economics back into economic geography. While I do not
wish to exclude all of mainstream economics out of hand, neither do
I wish to open the door to wholesale importation of mainstream economics.

Second, the mainstream makes a sharp distinction between microeconomics
and macroeconomics -- so much so that they sometimes seem to be entirely
separate. Nonetheless, mainstream macroeconomists has a strong affinity,
if not an absolute requirement for, ``microfoundations'' in macroeconomics
\parencite{wren-lewis2012microfounded}. This suggests, and in many
ways is required by, methodological individualism, which is still
prominent in mainstream economics \parencite{janssen2006microfoundations}.
\citeauthor{bhaskar2014thepossibility}'s \parencite*[29-34]{bhaskar2014thepossibility}
critique of methodological individualism -- and indeed of all ``one-way''
methodological approaches -- implies that this particular kind of
connection between the praxis and societal levels is illegitimate.
So what at first appears to be a bridge between micro and macro is
in fact a bridge falling down.

Third, in heterodox economics the micro- and macro- levels are at
once separate yet closely related. \textcite[3]{mearman2007teaching}
states a basic principle of heterodox economics: ``While theories
of the individual are uesful, so are theories of aggregate or collective
outcomes. Further, neither the individual nor the aggregate can be
understood in isolation from the other.'' This is entirely consistent
with \citeauthor{bhaskar2014thepossibility}'s methodological analysis,
which argues that the societal and praxis levels are distinct but
mutually interdependent. This is an extremely important but very subtle
difference between heterodox and mainstream economics, and this distinction
is too easily lost by using the mainstream labels.

Fourth, in economic geography the terms can easily be muddled even
further. As I am using the term here, ``societal'' refers to society
as a whole and to processes that actually exist at the level of society
as a whole. But there necessarily is some correlation between ``society
as a whole'' and geographic areas, such as nation states, continents,
or the global economy. So it is easy to use ``macroeconomic geography''
in a way that confuses and subtly elides societal processes (i.e.
an ontologically real stratum of the world at which society-wide occur)
and geographically large-scale processes. For example, economic crises
are inherently societal but need not be global, while international
trade is inherently global but need not be societal. For example,
\textcite{Peck_2016_Macroeconomic} labels as ``macroeconomic geography''
\citeauthor{hudson2016risingpowers}'s (\citeyear{hudson2016risingpowers})
paper on uneven global development, even though several of the global
processes \citeauthor{hudson2016risingpowers} discusses are not macroeconomic
(in the usual sense) while some macroeconomic processes that are not
geographically global, such as growing inequality and the decline
of union density in the U.S., have geographically global ramifications.
For similar reasons, I am tempted to use ``local'' instead of ``praxis''
as substitute labels for \citeauthor{bhaskar2014thepossibility}'s
``individual'' level, but ``local'' is too easily misinterpreted
as geographically local rather than as individual and immediate social
relations.}

This might be surprising, but consider Haila's \parencite*{haila1990thetheory}
seminal review of land rent theory between roughly 1970 and 1990.
She identifies three groups of questions around which she organizes
the literature:
\begin{enumerate}[label=\alph*)]
\item How does rent emerge?
\item Who or what are its agents, and what are their behavioral patterns
and social relations?
\item What is the economic role of rent?
\end{enumerate}
If microeconomics pertains to individuals, firms, etc., the second
question self-evidently falls within this realm. The first question
deals with the \emph{mechanisms} by which rent comes into being, and
to address this question the theory employs behavior and activities
of 

For example, \textcite[1, 21]{ward2016theshitty} give a thorough
history of land-rent theory, describing the literature as filled with
``confusion and conflict'' and needing rennovation, with heterodox
theory in particular having ``fallen into a state of dilapidation.''
They attribute this to three characteristic tendencies of geographic
political economy in the late 1980s. (1) Rejection of structuralism
undermined concern with land rent as a broad economic category and
shifted attention to the institutions and actors involved in processes
related to land rent, as well as to economic rents and fictitious
capital in general; (2) still heavily influenced Marxian political
economy, scholars deemphasized market prices in favor of labor values
and therefore neglected the bid-rent mechanism as a way to understand
land markets; and (3) confusion about absolute rent hobbled understanding
of the dynamics of not only land rent but also of other economic rents,
such as intellectual property rights, that were becoming increasinly
important in contemporary capitalism. As a way to move forward, \citeauthor{ward2016theshitty}
propose a broad, if sketchy, theoretical framework in which variations
in productivity and utility between land parcels create conditions
for differential rents, while supply shortages of particular kinds
of land create conditions for monopoly rents. They insinuate that
neoclassical theory, with its bid-rent construct, is the best candidate
for yielding insights into contermporary differential rents and that
institutional theory can serve the same purpose with monopoly rents.
A merger of these two approaches could then be redirected to investigate
``how capital flows through land,'' thereby avoiding the mistake
of conflating land and capital and exposing contradictions between
the two.

Others writing on the subject take a similar tack, assessing land-rent
theory's decrepit condition and suggesting how a revived theory could
help shed light on crucial contemporary issues. For example, \textcite{anderson2014classmonopoly}
notes that land rent is almost completely absent from the copious
literature on neoliberal cities and crisis and ascribes the ``loss
of interest'' in rent since the 1990s to the cultural turn in urban
studies, which pointed the field away from more economic topics. He
then argues that class monopoly rent could be very useful for understanding
neoliberal cities and their crises. Similarly, \textcite[136-137]{christophers2016forreal}
describes land (and land rent) as having been ``theoretically side-lined''
but attributes this to rent's status as ``fictitious capital'' in
both Marxian and Polanyian traditions: ``Faced with an array of real
and fictitious commodities and, in particular, of real and fictitious
capitals ..., is it any wonder that theorists have focused on the
real stuff and made the fictitious a secondary consideration?'' His
remedy, therefore, is to jetison the notion of fictitious capital
and to treat land as a real commodity.\footnote{Elsewhere in his essay \textcite[135]{christophers2016forreal} quotes
\textcite{elden2010landterrain}, who maintains that land ``is not
something that can be created but is a scarce resource.'' Accepting
this common assumption about land and still classifying land as a
commodity implies rejecting the usual Marxist practice of treating
``commodities'' as marketed products of labor. Below I argue that
land \emph{often is} a product of human labor, so that \citeauthor{christophers2016forreal}
and others are wrong about the supply of land being fixed, permanent,
and not humanly created and therefore also wrong to use land as justification
for broadening the concept of a ``commodity'' to include things
that are not products of human labor.}

\textcite{park2014landrent} is another example. After reviewing debates
concerning land rent before the period of ``rupture'' in the late
1980s, \textcite[99]{park2014landrent} describes the subsequent period
as one of ``premature stagnation'' and attributes this to four causes.\footnote{\textcite{haila1990thetheory} had described land-rent theory as having
gone through three phases: ``consensus'' during the 1970s, ``transition''
from the late 1970s to early 1980s, and ``rupture'' after the late
1980s. She also had interpreted land-rent theory as dealing with (1)
mechanisms of land-rent extraction, (2) social relations of property
and rent, and (3) the economic role of land rent. } One is only partial: land-rent mechanisms were neglected because
interest shifted to the social relations of land rent and to land
rent's economic role. Geographical political economy's critique and
rejection of neoclassical urban economics, coupled with the latter's
emphasis on land-rent mechanisms, only added to this neglect. A second
cause was failure to come even close to a consensus about absolute
rent, with differences of opinion about: the necessity of a low organic
composition of capital for absolute rent to exist; the distinction,
if any, between absolute rent and monopoly rent; historical conditions
for absolute rent to be important; and the possibility that something
resembling absolute rent could arise from transaction costs, monitoring
costs, and risk compensation. Third, confusion over the distinction
between rent paid for land versus rent paid for use of a building.
And fourth, the rupture itself, between what \citeauthor{haila1990thetheory}
had called ``ideographic'' and ``nomothetic'' rent theorists:
i.e., those who understand land rent as being highly contingent and
therefore requiring specific empirical study in concrete historical/institutional
contexts versus those who believe land rent undercapitalsm has a certain
general lawfulness. Interestingly, he points out the relevance recent
work on land rent in mainstream economics has for Marxian and other
heterodox theories of land rent. \citeauthor{park2014landrent} then
sketches a way forward through a research program that (1) develops
a consistent theory of land-rent mechanisms, (2) resolves issues surrounding
distinctions between absolute and monopoly rents, (3) identifies what
the ``product of the land'' means in an urban context, and (4) successfully
translates agrarian theories of land-rent to an urbanized economy. 

While these authors are correct in their diagnoses that land-rent
theory has been neglected and in a state of decline, their prescriptions
share one common feature: they all understand land-rent in the context
of what one might call ``microeconomic geography.''\footnote{With some trepedation, I borrow, or perhaps ape, mainstream economics'
distinction between microeconomic and macroeconomic. \textcite{Peck_2016_Macroeconomic}
uses ``macroeconomic geography'' to characterize Hudson's \parencite*{hudson2016risingpowers}
account of global economic change under neoliberalism. But here ``macro''
signals large geographic scale (global or continental as opposed to
neighborhood, local, or perhps regional), whereas in economics it
distinguishes collective, economy-wide from individual or even industry-wide
economic processes.}

\paragraph*{For opening paragraph}

\cite{lainton2013rethinking}

\cite{lainton2013theclassical}

\paragraph*{For second paragraph - Macro}

\cite{beitel2016circuits,florida4thereal}

\paragraph*{For elsewhere}

\cite{balardini1demandand}

\cite{emmett1frankh}

\cite{foldvary1themarginalists}

\cite{gaffney2008keeping}

\cite{goodman2008whereare}

\cite{munro2012landand}

\cite{williams2010bichler}

\cite{sungyoonkang2010information}

\cite{ecobomakis2003onabsolute}

\cite{ramirez06marxstheory}

\cite{vercellone2008thenew}

\begin{comment}
Revision up to here
\end{comment}

A recent review of the history of land-rent theory characterize it
as filled with ``confusion''

``c\slash onfusion and conflict,'' and particularly heterodox land-rent
theory having ``fallen into a state of dilapidation'' and needing
renovation \parencite[1, 21]{ward2016theshitty}.

'' throughout its history. In particular, they single out contemporary
as \nocite{ward2016theshitty}(p. 21) and in need of renovation .
Based on their analysis of the reasons for this decline, they briefly
offer a blueprint fqor reviving land-rent theory and encourage debate
around this project.

In this they are not alone. Several other scholars also have recently
called attention to land-rent theory's decrepit condition and offered
prescriptions for its renewal. But although they share a general consensus
regarding land-rent theory's poor condition and the desirability of
reviving it, they vary both in their diagnoses of what is wrong with
the theory and in their proposals for reviving it. In some instances,
such differences may be reconcilable, or even minor, with complementary-but-different
positions on the subject readily capable of being combined: it is
a matter of ``both/and.'' In other cases, possibilities for reconciliation
are not ob\slash vious and different positions seem to be starkly
opposed: ``either/or'' appears to be the only possibility. In such
instances, engagement and debate might resolve conflict between positions,
or at least bring to light why such resolution is impossible. Perhaps
such debate would result in dropping one approach in favor of another.
At the very least it could highlight strengths and weaknesses in different
positions. Yet among the various proposals currently on the table,
I know of no such debates. Instead, a more ``tribal'' pattern of
disengagement and isolation seems likely, with different approaches
going their own ways and ignoring the others.\footnote{This conclusion follows from a particular reading of the history of
land-rent theory in the larger context of economic geography. See
Section \ref{sec:History}.}

To some extent the present essay hopes to forestall this by insisting
on \emph{theoretical coherence}: not only in land-rent theory itself
but in conjunction with the larger body of heterodox political economy,
including both geographical political economy and heterodox economics.
In other words, the essay rethinks land-rent theory in several ways,
all of which potentially link land-rent theory to broader currents
of contemporary heterodox political-economy. This is not to say that
this is the only legitimate way to theorize land-rent: other ways
are certainly possible and may yield important insights. But a trans-disciplinary
political-economic approach can be an extraordinarily powerful tool,
and the challenge for advocates of alternative approaches is to demonstrate
that any alternative is at least equally powerful. It may be that
a specific theory of land-rent is better able to explain an isolated
case, particularly if the theory employs \emph{ad hoc} concepts tailored
to the case. But the requirement of broader theoretical coherence
implies that any specific land-rent theory must be consistent with
and part of a larger theoretical corpus and that the evaluation of
competing theories of land rent must involve the larger corpus as
a whole.\footnote{This is not to say that two or more competing theories of land rent
cannot \emph{both} be consistent with a \emph{single}, broader theoretical
framework. Such possibilities are always open. But in such cases,
adjudication between the competitors cannot be confined solely to
land rent itself and must instead evaluate the competitors \emph{as
part of} the larger framework. Similarly, a single element in land-rent
theory may be compatible within two or more broader frameworks. In
such cases, rather than just letting the land-rent component stand
by itself or arbitrarily choosing one of the larger frameworks, evaluation
of alternatives also requires evaluating the single land-rent theory
in the context of the competing, larger frameworks. }

\subsection{Motivating Concerns}

Several concerns motivate this approach. Most fundamental is the fragmentation
of economic geography since the early 1990s: growing fragmentation
over the past two-and-a-half decades contributed to land-rent theory's
decline three ways. First, to the extent to which land-rent theory
was present in a corpus of scholarship in which different tendencies
associated with similar tendencies in the larger body of economic
geography scholarship, fragmentation in the latter tended to \emph{pulverize}
land-rent theory, shattering whatever wholeness and cohesiveness it
might have had. Second, pulverization of land-rent theory encouraged
different approaches to become isolated from each other, and this
in turn allowed them to \emph{immunize} themselves from external criticism:
a given approach could become self-referential and give itself license
to ignore alternative approaches that might otherwise be its competitors
or critics. Fragmentation also \emph{impoverished} land-rent theory
in a number of ways.

\begin{comment}
Revision up to here
\end{comment}

It undermined land-rent theory's cohesiveness, and it isolated different
approaches to economic geography from each other and from important
strands of heterodox scholarship in other social science disciplines.
Section \ref{sec:History} below, discusses the history of economic
geography and land-rent theory and explains how economic geography's
fragmentation eroded land-rent theory's cohesiveness. As for isolation,
it had two closely related, detrimental effects. %
\begin{comment}
Maybe cite Aalbers \& Christophers on housing.
\end{comment}

On one hand, fragmentation-\emph{cum}-isolation allows self-referential,
safe-spaces for scholarship that may seem valid within a self-selected
intellectual cocoon, but that is obviously invalid, at least without
major revision, when considered in a larger intellectual context.
This is not mainly a matter of different paradigms: although debates
between paradigms can be fruitful and therefore desirable, they are
extremely complex and difficult because paradigmatic differences are
so very basic. Instead, within \emph{a common }paradigm, such intellectual
siloing can prevent incompatible theoretical interpretations from
having to confront each other, and thereby prevent each from being
supplanted by the other, combined in a new synthesis, or exposed as
requiring reconsideration and modification in order to be consistent
with other, related bodies of scholarship.\footnote{Interestingly, the \emph{Journal of Economic Geography} was established
in 2001 with the express goal of encouraging interchange between economists
and economic geographers. But economists contributing to the journal
have overwhelmingly been mainstream economists, while economic geographers
mostly have heterodox leanings \cite{sheppard2011geographical}. It
is therefore unsurprising that in the pages of the journal one can
see sharp divisions between the two disciplines and seemingly insurmountable
obstacles in a cross-paradigmatic \foreignlanguage{american}{\emph{methodenstreit}}
\cite{garretsen2011thejournal}. With some irony, heterodox economists
-- who are far more likely to have productive interaction with today's
economic geographers -- have largely been absent from the journal.
Given how little explicit geography one finds in contemporary heterodox
economics, this is unsurprising, but it also may reflect the predispositions
of the journal's economist reviewers to act as gatekeepers, confining
participation by the economics profession to the mainstream.}

On the other hand, even when a particular theoretical intervention
is valid on its own terms, siloing cuts it off from other, complementary
threads within the same theoretical framework. This impoverishes the
larger framework and all its components. Breaking down such walls
between ``tribes'' in today's economic geography and between geographical
political economy itself and related heterodox scholarship in other
social science disciplines has great potential to enrich all of contemporary
heterodox political economy. Section \ref{sec:Examples}has several
examples of how land-rent theory might fruitfully combine with heterodox
scholarship focused on other topics.

\begin{comment}
Continue here.
\end{comment}

Another concern stems from the Crisis of the Early Twenty-First Century,
or what for the sake of brevity I shall refer to here as ``Crisis-21.''\footnote{Use the footnote from earlier drafts.}

A third concern is closely related to the crisis and the inadequacy
of most analyses of it with regard to the role of land-rent: land-rent
theory's preoccupation with micro- and meso-level mechanisms of rent
extraction and distribution and its almost complete neglect of land
rent as a macroeconomic and macro-geographical process.

\printbibliography


\end{document}
